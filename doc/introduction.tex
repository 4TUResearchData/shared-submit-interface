\chapter{Introduction}

\section{Running the \t{shared-submit-interface}.}

\subsection{Configuring the database back-end}

The \t{shared-submit-interface} can be used with a SPARQL 1.1
compliant database.  The authors tested with both Virtuoso Open Source
and BerkeleyDB.

\subsubsection{Configuring \t{berkeleydb} as back-end}

The simplest setup can be achieved by leveraging RDFLib's
stores.  One such option is the \t{berkeleydb} back-end. To use this option,
ensure that both the \t{berkeleydb} system libraries and the \t{berkeleydb}
Python package are installed.

To use the \t{berkeleydb} back-end, configure the \t{rdf-store} as following:
\begin{lstlisting}[language=xml]
<rdf-store>
  <sparql-uri>(@*\Highlight{bdb://}*@)/path/to/empty/directory</sparql-uri>
  <state-graph>ssi://default</state-graph>
</rdf-store>
\end{lstlisting}

\subsubsection{Configuring \t{virtuoso} as back-end}

A higher performant and more scalable setup than \t{berkeleydb} can be
achieved by using Virtuoso Open Source.

To use the \t{virtuoso} back-end, configure the \t{rdf-store} as following:
\begin{lstlisting}[language=xml]
<rdf-store>
  <sparql-uri>(@*\Highlight{https://adress-of-endpoint:8890/sparql}*@)</sparql-uri>
  <state-graph>ssi://default</state-graph>
</rdf-store>
\end{lstlisting}

This assumes the \t{virtuoso} endpoint is available at
\t{https://address-of-endpoint:8890/sparql}.

