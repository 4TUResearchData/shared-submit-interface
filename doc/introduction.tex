\chapter{Introduction}

\section{Running the \t{shared-submit-interface}.}

\subsection{Configuring the database back-end}

The \t{shared-submit-interface} can be used with a SPARQL 1.1
compliant database.  The authors tested with both Virtuoso Open Source
and BerkeleyDB.

\subsubsection{Configuring \t{berkeleydb} as back-end}

The simplest setup can be achieved by leveraging RDFLib's
stores.  One such option is the \t{berkeleydb} back-end. To use this option,
ensure that both the \t{berkeleydb} system libraries and the \t{berkeleydb}
Python package are installed.

To use the \t{berkeleydb} back-end, configure the \t{rdf-store} as following:
\begin{lstlisting}[language=xml]
<rdf-store>
  <sparql-uri>(@*\Highlight{bdb://}*@)/path/to/empty/directory</sparql-uri>
  <state-graph>ssi://default</state-graph>
</rdf-store>
\end{lstlisting}

\subsubsection{Configuring \t{virtuoso} as back-end}

A higher performant and more scalable setup than \t{berkeleydb} can be
achieved by using Virtuoso Open Source.

To use the \t{virtuoso} back-end, configure the \t{rdf-store} as following:
\begin{lstlisting}[language=xml]
<rdf-store>
  <sparql-uri>(@*\Highlight{https://adress-of-endpoint:8890/sparql}*@)</sparql-uri>
  <state-graph>ssi://default</state-graph>
</rdf-store>
\end{lstlisting}

This assumes the \t{virtuoso} endpoint is available at
\t{https://address-of-endpoint:8890/sparql}.

\section{Application flow}

The \t{shared-submit-interface} implements a three-step process to route a
user to a data repository.

\subsection{Step 1: Authenticate using SURF}

The first of these three steps is logging in using SURF Research Access
Management (SRAM).  By doing so, the user's authenticity can be guaranteed
when relaying information to the data repositories.

\subsection{Step 2: Gathering metadata}

The second step involves entering metadata that is relevant to determining
the data repository, and additionally, for the user to identify the dataset
once it's transfered to a data repository.

Uploading data has been left out of the scope, because it would complicate
the setup, and handling of sensitive data would be a disaster.

\subsection{Step 3: Transfering metadata to the designated repository}

The final step in the process transfers the metadata to the designated
data repository.  The \t{shared-submit-interface} sends the following
data using a HTTP PUT request that the data repository needs to
implement:

Example content:
\begin{lstlisting}[language=json]
  {
    "psk": "--- The configured pre-shared key for this endpoint ---"
    "email": "--- E-mail address provided by SURF. ---",
    "title": "Example title",
    "domain": "Physical and Technical Sciences",
    "affiliation": "Delft University of Technology",
    "datatype": "software"
  }
\end{lstlisting}

A succesful execution is assumed if the data repository responds with
a HTTP 302 (redirect).  This redirect is expected to lead to the
submission form of the data repository.
